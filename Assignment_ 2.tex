\documentclass[journal,12pt,twocolumn]{IEEEtran}

\usepackage{setspace}
\usepackage{gensymb}

\singlespacing


\usepackage[cmex10]{amsmath}
%\usepackage{amsthm}
%\interdisplaylinepenalty=2500
%\savesymbol{iint}
%\usepackage{txfonts}
%\restoresymbol{TXF}{iint}
%\usepackage{wasysym}
\usepackage{amsthm}
%\usepackage{iithtlc}
\usepackage{mathrsfs}
\usepackage{txfonts}
\usepackage{stfloats}
\usepackage{bm}
\usepackage{cite}
\usepackage{cases}
\usepackage{subfig}
%\usepackage{xtab}
\usepackage{longtable}
\usepackage{multirow}
%\usepackage{algorithm}
%\usepackage{algpseudocode}
\usepackage{enumitem}
\usepackage{mathtools}
\usepackage{graphicx}
\usepackage{refstyle}
\usepackage{caption}
\usepackage{steinmetz}
\usepackage{tikz}
%\usepackage{circuitikz}
\usepackage{verbatim}
\usepackage{tfrupee}
\usepackage[breaklinks=true]{hyperref}
%\usepackage{stmaryrd}
\usepackage{tkz-euclide} % loads  TikZ and tkz-base
%\usetkzobj{all}
\usetikzlibrary{calc,math}
\usepackage{listings}
   \usepackage{color}                                            %%
    \usepackage{array}                                            %%
    \usepackage{longtable}                                        %%
    \usepackage{calc}                                             %%
    \usepackage{multirow}                                         %%
    \usepackage{hhline}                                           %%
    \usepackage{ifthen}                                           %%
  %optionally (for landscape tables embedded in another document): %%
    \usepackage{lscape}     
%\usepackage{multicol}
\usepackage{chngcntr}
%\usepackage{enumerate}

%\usepackage{wasysym}
%\newcounter{MYtempeqncnt}
\DeclareMathOperator*{\Res}{Res}
%\renewcommand{\baselinestretch}{2}
\renewcommand\thesection{\arabic{section}}
\renewcommand\thesubsection{\thesection.\arabic{subsection}}
\renewcommand\thesubsubsection{\thesubsection.\arabic{subsubsection}}

\renewcommand\thesectiondis{\arabic{section}}
\renewcommand\thesubsectiondis{\thesectiondis.\arabic{subsection}}
\renewcommand\thesubsubsectiondis{\thesubsectiondis.\arabic{subsubsection}}

% correct bad hyphenation here
\hyphenation{op-tical net-works semi-conduc-tor}
\def\inputGnumericTable{}                                 %%

\lstset{
%language=C,
frame=single, 
breaklines=true,
columns=fullflexible
}

\begin{document}

\newtheorem{theorem}{Theorem}[section]
\newtheorem{problem}{Problem}
\newtheorem{proposition}{Proposition}[section]
\newtheorem{lemma}{Lemma}[section]
\newtheorem{corollary}[theorem]{Corollary}
\newtheorem{example}{Example}[section]
\newtheorem{definition}[problem]{Definition}

\newcommand{\BEQA}{\begin{eqnarray}}
\newcommand{\EEQA}{\end{eqnarray}}
\newcommand{\define}{\stackrel{\triangle}{=}}
\bibliographystyle{IEEEtran}
%\bibliographystyle{ieeetr}
\providecommand{\mbf}{\mathbf}
\providecommand{\pr}[1]{\ensuremath{\Pr\left(#1\right)}}
\providecommand{\qfunc}[1]{\ensuremath{Q\left(#1\right)}}
\providecommand{\sbrak}[1]{\ensuremath{{}\left[#1\right]}}
\providecommand{\lsbrak}[1]{\ensuremath{{}\left[#1\right.}}
\providecommand{\rsbrak}[1]{\ensuremath{{}\left.#1\right]}}
\providecommand{\brak}[1]{\ensuremath{\left(#1\right)}}
\providecommand{\lbrak}[1]{\ensuremath{\left(#1\right.}}
\providecommand{\rbrak}[1]{\ensuremath{\left.#1\right)}}
\providecommand{\cbrak}[1]{\ensuremath{\left\{#1\right\}}}
\providecommand{\lcbrak}[1]{\ensuremath{\left\{#1\right.}}
\providecommand{\rcbrak}[1]{\ensuremath{\left.#1\right\}}}
\theoremstyle{remark}
\newtheorem{rem}{Remark}
\newcommand{\sgn}{\mathop{\mathrm{sgn}}}
%\providecommand{\abs}[1]{\left\vert#1\right\vert}
\providecommand{\res}[1]{\Res\displaylimits_{#1}} 
\providecommand{\norm}[1]{\left\lVert#1\right\rVert}
\providecommand{\norm}[1]{\lVert#1\rVert}
\providecommand{\mtx}[1]{\mathbf{#1}}
%\providecommand{\mean}[1]{E\left[ #1 \right]}
\providecommand{\fourier}{\overset{\mathcal{F}}{ \rightleftharpoons}}
%\providecommand{\hilbert}{\overset{\mathcal{H}}{ \rightleftharpoons}}
\providecommand{\system}{\overset{\mathcal{H}}{ \longleftrightarrow}}
	%\newcommand{\solution}[2]{\textbf{Solution:}{#1}}
\newcommand{\solution}{\noindent \textbf{Solution: }}
\newcommand{\cosec}{\,\text{cosec}\,}
\providecommand{\dec}[2]{\ensuremath{\overset{#1}{\underset{#2}{\gtrless}}}}
\newcommand{\myvec}[1]{\ensuremath{\begin{pmatrix}#1\end{pmatrix}}}
\newcommand{\mydet}[1]{\ensuremath{\begin{vmatrix}#1\end{vmatrix}}}
%\numberwithin{equation}{section}
\numberwithin{equation}{subsection}
%\numberwithin{problem}{section}
%\numberwithin{definition}{section}
\makeatletter
\@addtoreset{figure}{problem}
\makeatother
\let\StandardTheFigure\thefigure
\let\vec\mathbf
%\renewcommand{\thefigure}{\theproblem.\arabic{figure}}
\renewcommand{\thefigure}{\theproblem}
%\setlist[enumerate,1]{before=\renewcommand\theequation{\theenumi.\arabic{equation}}
%\counterwithin{equation}{enumi}
%\renewcommand{\theequation}{\arabic{subsection}.\arabic{equation}}
\def\putbox#1#2#3{\makebox[0in][l]{\makebox[#1][l]{}\raisebox{\baselineskip}[0in][0in]{\raisebox{#2}[0in][0in]{#3}}}}
     \def\rightbox#1{\makebox[0in][r]{#1}}
     \def\centbox#1{\makebox[0in]{#1}}
     \def\topbox#1{\raisebox{-\baselineskip}[0in][0in]{#1}}
     \def\midbox#1{\raisebox{-0.5\baselineskip}[0in][0in]{#1}}
\vspace{3cm}
\title{Assignment-2}
\author{Satyam Singh \\ EE20MTECH14015}
\maketitle
\newpage
\bigskip
\renewcommand{\thefigure}{\theenumi}
\renewcommand{\thetable}{\theenumi}
\begin{abstract}
This assignment finds the equation of the two straight lines from second degree  equation.
\end{abstract}
Download all python codes from 
\begin{lstlisting}
https://github.com/satyam463/matrix-theory/blob/master/assignment2.py
\end{lstlisting}
\section{Problem Statement}
Prove that the following equations represents two straight lines also find their point of intersection and angle between them.
\begin{align}
y^2+xy-2x^2-5x-y-2=0
\end{align}
\section{Solution}
The general equation of second degree equation is given by
\begin{align}
ax^2+2bxy+cy^2+2dx+2ey+f=0
\end{align}
and can be expressed as
\begin{align}
\vec{x}^T\vec{V}\vec{x}+2\vec{u}^T\vec{x}+f=0 \\
\vec{V}=\vec{V}^T=\myvec{a & b \\ b & c} ;
\vec{u}=\myvec{d & e}
\end{align}
represents a pair of straight lines if
\begin{align}
\mydet{
\vec{V}&\vec{u}^T\\
\vec{u}&f} = 0
\end{align}
On comparing equation(1.0.1) and equation(2.0.1) we get
\begin{align}
a=-2,b=\frac{1}{2},c=1,d=\frac{-5}{2},e=\frac{-1}{2},f=-2
\end{align}
substituting above value in equation(2.0.4) we get
\begin{align}
\mydet{
-2&\frac{1}{2}&\frac{-5}{2}\\
\frac{1}{2}&1&\frac{-1}{2}\\
\frac{-5}{2}&\frac{-1}{2}&-2}
\xleftrightarrow[R_1\rightarrow R_1+R_3]{R_1\rightarrow R_1-R_2}
\mydet{
0&0&0\\
\frac{1}{2}&1&\frac{-1}{2}\\
\frac{-5}{2}&\frac{-1}{2}&-2}=0
\end{align}
Hence this represents pair of  two straight lines.
Now differentiating partially the \equationautorefname{1.0.1} w.r.t y and x
\begin{align}
    \myvec{1&2}\vec{x} = 1\\
    \myvec{-4&1}\vec{x}=5
\end{align}
consider augmented matrix
\begin{align}
    \myvec{1&2&1\\-4&1&5}
    \xleftrightarrow[R_2\rightarrow R_2/9]{R_2\rightarrow R_2+4R_1}
    \myvec{1&2&1\\0&1&1}\xleftrightarrow{R_1\rightarrow R_1-2R_2}
    \myvec{1&0&-1\\0&1&1}
\end{align}
Therefore point of intersection is $\vec{A}=\myvec{-1\\1}.$
Now Angle between two lines can be expressed in terms of normal vector
\begin{align}
\vec{n_1}=\myvec{1\\2}\text{ and }\vec{n_2}=\myvec{-4\\1}
\end{align}
Angle between two lines $\theta$ can be given by
\begin{align}
\cos\theta =\frac{\vec{n_1}^T\vec{n_2}}{\norm{\vec{n_1}}\norm{\vec{n_2}}}=
\frac{\myvec{1&2}\myvec{-4\\1}}{\sqrt{\left(2\right)^2 +1} \times \sqrt{(-4)^2 +1}}=\frac{-2}{\sqrt{85}}\\
\theta = \cos^{-1}(\frac{-2}{\sqrt{85}})
\end{align}
\begin{figure}[!ht]
\centering
\includegraphics[width=\columnwidth]{liner.png}
\caption{plot showing intersection of lines}
\label{Fig}
\end{figure}
\end{document}
