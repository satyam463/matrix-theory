\documentclass{article}
\usepackage[utf8]{inputenc}
\usepackage{mathtools}
\usepackage{bm}
\usepackage{amsfonts}
\usepackage{amsmath}
\usepackage{amssymb}
\let\oldvec=\vec
\renewcommand{\vec}[1]{\mathbf{#1}}
\usepackage{refstyle}
\usepackage{amsthm}
\usepackage{geometry}
\usepackage{graphicx}
\usepackage{enumitem}
\newcommand{\myvec}[1]{\ensuremath{\begin{pmatrix}#1\end{pmatrix}}}
\begin{document}
\section{Problem Statement}
Find the equation of the line parallel to the Y-axis drawn through the point of intersection of the lines
\begin{align}
\myvec{1 & -7}\Vec{x}  =-5 \\ \myvec{3 & 1}\Vec{x}= 0
\end{align}
\section{Theory}
consider the equation of the system of lines
\begin{align}
x - 7y & = -5 \\
3x + y & = 0
\end{align}
 consider the augmented matrix
 \begin{align}
 \myvec{1 & -7 & -5 \\ 3 & 1 & 0}
 \end{align}
 By applying row reduction reduction technique 
 \begin{align}
\myvec{4 & -7 & -5\\ 3 & 1 & 0}
	\xleftrightarrow[R_2 \rightarrow R_2/22]{R_2 \rightarrow R_2-3R_1}
	\myvec{1 & -7 & -5 \\ 0 & 1 & \frac{15}{22}}
	\xrightarrow[]{R_1 \rightarrow R_1+7R_2}
	 \myvec{1 & 0 & \frac{-5}{22} \\ 0 & 1 & \frac{15}{22}}
 \end{align}
 The value of $\Vec{A}$ is the point of intersection.
\begin{align}
 {\Vec{A}} = \myvec{\frac{-5}{22} \\ \frac{15}{22}} ;
\end{align}
Now the equation of line parallel to y-axis through the point of intersection
\begin{align}
\Vec{n}^T(\Vec{x}-\Vec{A}) = 0
\end{align}
where$\Vec{n}$ is the vector normal to the Y - axis and $\vec{A}$ is the point of intersection.\\
\begin{align*}
\vec{n}^T \Vec{x} = \Vec{n}^T\vec{A} ; where
 \Vec{n}^T = \myvec{1 & 0}
 \end{align*}
 \begin{align}
\myvec{1 & 0}\vec{x} &  = \myvec{1 & 0} \myvec{\frac{-5}{22} \\ \frac{15}{22}}\\
\vec{x} & = \myvec{\frac{-5}{22}} 
 \end{align}
 \begin{figure}
 \centering
 \includegraphics[width=\columnwidth]{line.png}
 \caption{graphical representation of systems of lines}
 \label{fig:lines}
 Shown in \figref{lines} is the equation of the line parallel to the Y-axis drawn through the point of intersection of the lines 
 \end{figure}
\end{document}
